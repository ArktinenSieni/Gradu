%% This file is modified by Veli Mäkinen from HY_fysiikka_LuKtemplate.tex authored by Roope Halonen ja Tomi Vainio.
%% Some text is also inherited from engl_malli.tex by Kutvonen, Erkiö, Mäkelä, Verkamo, Kurhila, and Nykänen.


% STEP 1: Choose oneside or twoside
\documentclass[finnish,twoside,openright]{HYgraduMLDS}
%finnish,swedish

%\usepackage[utf8]{inputenc} % For UTF8 support. Use UTF8 when saving your file.
\usepackage{lmodern} % Font package
\usepackage{textcomp} % Package for special symbols
\usepackage[pdftex]{color, graphicx} % For pdf output and jpg/png graphics
\usepackage[pdftex, plainpages=false]{hyperref} % For hyperlinks and pdf metadata
\usepackage{fancyhdr} % For nicer page headers
\usepackage{tikz} % For making vector graphics (hard to learn but powerful)
%\usepackage{wrapfig} % For nice text-wrapping figures (use at own discretion)
\usepackage{amsmath, amssymb} % For better math
%\usepackage[square]{natbib} % For bibliography
\usepackage[footnotesize,bf]{caption} % For more control over figure captions
\usepackage{blindtext}
\usepackage{titlesec}
\usepackage[titletoc]{appendix}
\usepackage{hyperref}

\onehalfspacing %line spacing
%\singlespacing
%\doublespacing

%\fussy 
\sloppy % sloppy and fussy commands can be used to avoid overlong text lines

% STEP 2:
% Set up all the information for the title page and the abstract form.
% Replace parameters with your information.
\title{SQL:n Oppiminen Datatieteen Menetelmin}
\author{Matti Räty}
\date{\today}
\prof{Professor X or Dr. Y}
\censors{Professor A}{Dr. B}{}
\keywords{layout, summary, list of references}
\depositeplace{}
\additionalinformation{}


\classification{\protect{\ \\
\  General and reference $\rightarrow$ Document types  $\rightarrow$ Surveys and overviews\  \\
\  Applied computing  $\rightarrow$ Document management and text processing  $\rightarrow$ Document management $\rightarrow$ Text editing\\
}}

% if you want to quote someone special. You can comment this line and there will be nothing on the document.
%\quoting{Bachelor's degrees make pretty good placemats if you get them laminated.}{Jeph Jacques} 


% OPTIONAL STEP: Set up properties and metadata for the pdf file that pdfLaTeX makes.
% But you don't really need to do this unless you want to.
\hypersetup{
    bookmarks=true,         % show bookmarks bar first?
    unicode=true,           % to show non-Latin characters in Acrobat’s bookmarks
    pdftoolbar=true,        % show Acrobat’s toolbar?
    pdfmenubar=true,        % show Acrobat’s menu?
    pdffitwindow=false,     % window fit to page when opened
    pdfstartview={FitH},    % fits the width of the page to the window
    pdftitle={},            % title
    pdfauthor={},           % author
    pdfsubject={},          % subject of the document
    pdfcreator={},          % creator of the document
    pdfproducer={pdfLaTeX}, % producer of the document
    pdfkeywords={something} {something else}, % list of keywords for
    pdfnewwindow=true,      % links in new window
    colorlinks=true,        % false: boxed links; true: colored links
    linkcolor=black,        % color of internal links
    citecolor=black,        % color of links to bibliography
    filecolor=magenta,      % color of file links
    urlcolor=cyan           % color of external links
}

\begin{document}

% Generate title page.
\maketitle

% STEP 3:
% Write your abstract (of course you really do this last).
% You can make several abstract pages (if you want it in different languages),
% but you should also then redefine some of the above parameters in the proper
% language as well, in between the abstract definitions.

\begin{abstract}
Summary of the main contents of the work: topic, methodology and results.

Topics are classified according to the ACM Computing Classification System
(CCS): check command \verb+\classification{}+. A small set of paths (1-3) should be used, starting from any top nodes
referred to bu the root term CCS leading to the leaf nodes. The elements
in the path are separated by right arrow, and emphasis of each element individually can be indicated
by the use of bold face for high importance or italics for intermediate
level. The combination of individual boldface terms may give the reader
additional insight. 
\end{abstract}

% Place ToC
\mytableofcontents

\mynomenclature

% -----------------------------------------------------------------------------------
% STEP 4: Write the thesis.
% Your actual text starts here. You shouldn't mess with the code above the line except
% to change the parameters. Removing the abstract and ToC commands will mess up stuff.
\chapter{Johdanto}

The thesis should have an introduction chapter. Other chapters can be named according to the topic. In the end, some summary chapter is needed; see Chapter~\ref{chapter:Loppusanat} for an example.

\section{Tutkimuskysymykset}

\begin{itemize}
    \item Miten SQL:n oppimista on tutkittu?
    \begin{itemize}
        \item kirjallisuus
    \end{itemize}
    \item Millä tavoin kurssin tehtävien perusteella ennustaa oppimenestystä?
    \begin{itemize}
        \item Kirjallisuus
    \end{itemize}
    \item Millä tavoin SQL-kurssin oppimenestystä ennustaa kurssin tehtävien perusteella?
    \begin{itemize}
        \item kirjallisuus
        \item tämä data
    \end{itemize}
\end{itemize}


\chapter{Taustaa}

\section{SQL:n oppiminen}

Ainakin kolme artikkelia aiheesta.

\subsection{SQL Opetusohjelmistot}


\subsection{Tutkimusta SQL:n oppimisesta}


\section{Kurssi-menestyksen ennustaminen ohjelmointikursseilla}

Ainakin kolme artikkelia aiheesta.

\begin{itemize}
    \item Miksi kiinnostavaa
    \item Miten toteutettu, pari keissiä
    \begin{itemize}
        \item Mitä dataa ovat ottaneet
        \item Millä menetelmin
        \item Millaisia tuloksia
    \end{itemize}
\end{itemize}

Akateemisella menestyksellä tarkoitan yleisesti oppimenestystä, ja taitoja mitkä auttavat oppimisessa.

Kiinnostavia aiheita akatemmisen menestyksen ennustamisessa ovat esimerkiksi lisäapua vaativien oppilaiden tunnistaminen ja vaikeiden käsitteiden löytäminen automaattisesti.

Oppimenestyksen ennustamista voidaan tehdä usealla tasolla. Esimerkiksi yksittäistä kurssia, tehtävää tai oppilasta tarkastellen\cite{hellas2018predicting}. Tässä keskitytään yksittäisen kurssin ennustamiseen, joka on pidetty useampana vuonna. Ennustaminen tässä kontekstissa tehdään tehtyjen tehtävien perusteella.

Oppimenestyksen ennustamisen menetelmät ovat jaettavissa luokitteluun, data-louhintaan, tilastollisiin mentelmiin ja klusterointiin\cite{hellas2018predicting}. Käytetyimpiä koneoppimismalleja akateemisen menestyksen ennustamiseen ovat lineaariset regressiomallit ja luokittelumenetelmät. Esimerkkinä monimutkaisemmasta mallista on transferenssi-oppiminen (transfer learning)\cite{lagus2018transfer} ja nönnönnöö.



\section{SQL-kurssien menestyksen ennustaminen}

Ainakin kolme artikkelia aiheesta.


\chapter{Koneoppimisen soveltaminen kurssi-dataan}

\section{Data}

Mitä dataa, mistä on saatu ja suunnitelmia käytöstä.


\section{Valitut mentelmät}


\chapter{SQL-kurssin oppimenestyksen ennustamisen toteutus}

\section{Ohjelmakoodi}


\section{Prosessi, haasteet, yms.}


\section{Tulokset}

\chapter{Muistiinpanoja}

\section{Muistiinpanoja: The Normalized Programming State Model}

Oleellinen termi: explained variance
\begin{itemize}
    \item \url{https://en.m.wikipedia.org/wiki/Explained\_variation}
    \item \url{http://www.csun.edu/~mr31841/documents/Varitionandpredictionintervals.pdf}
\end{itemize}

\subsection{tutkimuskysymys}

\begin{itemize}
    \item Miten eri tavoin eri ympäristöt, kielet ja ohjelmointiympäristöt vaikuttavat Error Quotient\cite{jadud2006methods}- ja Watwin\cite{watson2014no} piste-asteikon ennustaviin kykyihin?
    \item Kuinka hyvin kokonaisvaltaisempi malli ennustaa suoritusta?
\end{itemize}


\subsection{tutkimustulokset}

Normalized Programming State Model eli NPSM, suoriutui tutkimuksen\cite{carter2015normalized} ympäristössä Error Qutient- ja Watwinin piste-asteikkoa paremmin.

\begin{itemize}
    \item RQ1
        \begin{itemize}
            \item rajua vaihtelua mittauksissa Error Quotient ja Watwin-asteikossa, 6.1:1:1
            \begin{itemize}
                \item Tämän kokeen asetelmassa suurempi arvosanapainotus tehtäville, 6.1:2:5
                \item edellämainitut tukeutuvat vahvemmin ohjelmointiympäristön virheviesteihin ja tukeen, ja Javan parempiin virheviesteihin. Verrattuna tämän tutkimuksen kieleen C++:n, 6.1:3:3
                \item Molemmat rankaisevat toistuvista virheviesteistä, 6.1:3:9
                \begin{itemize}
                    \item C++:n virheviestit voivat sisältää samoja virkkeitä eri virheissä -> false positive
                \end{itemize}
            \end{itemize}
            \item Tutkimuksessa toistui aikaisemmat ilmiöt: Error quotient toimii parhaiten pienemmillä tietolähteillä, ja Watwin suuremmilla tietolähteillä, 6.1:5:2
            \item Watwin-asteikon "variance explained" toistuu muissa tutkimuksissa, missä taas Error Quotient:n vastaava arvo ei toistu, 6.1:5:7
            \item Kaikki mallit vaativat lisätutkimuksia eri ympäristöissä, 6.1:5:13
        \end{itemize}
    \item RQ2
    \begin{itemize}
        \item NPSM päihitti mallit jotka keskittyivät vain kääntämis-käyttäytymiseen (compilation), 6.2:1
        \begin{itemize}
            \item Yksittäisissä tehtävissä Watwin asteikkoon nähden nelinkertaiset tulokset varianssin selitettävyydessä, ja Error Quotient:iin nähden prosentin verran
            \item Tehtävä-arvosanojen keskiarvossa 4x Watwin-asteikkoon, ja 6x Error Quotient asteikkoon nähden
            \item Loppuarvosanoijen ennustettavuudessa, 3x Watwin-asteikkoon, ja 12x Error Quotient nähden
        \end{itemize}
        \item Ajaminen semanttisesti väärää ohjelmaa, ja semanttisesti tuntematonta sovellusta edistivät onnistumista, 6.2:2:2
        \item tuntematon tila (eli oletustila ennen ensimmäistä ajoa) ja syntaktisesti väärän ohjelman ajaminen tekivät hallaa onnistumiselle, 6.2:2:6
        \item vetivät johtopäätöksen että sovelluksen ajaminen, oli kyseessä oikea ratkaisu tai ei, vaikuttaa olevan onnistuva ohjelmointi lähestymistapa, 6.2:2:11
        \item koodin kirjoittaminen ilman ajamista isoissa paloissa, päinvastoin, 6.2:2:13
    \end{itemize}
\end{itemize} 


\subsection{muistiinpanoja}

\begin{itemize}
    \item Ennusteet auttavat opettajia tunnistamaan riskirajoilla olevia opiskelijoita, ja parantamaan opetusmenetelmiä\cite{carter2015normalized}, Abstract:1:3
    \item Error Quotient asteikko\cite{jadud2006methods} ja Watwinin pisteasteikko\cite{watson2014no} ovat saavuttaneet menestystä tarkastellen ainoastaan opiskelijoiden kääntämis-yrityksiä(compilation), Abstract:1:6
    \item dataa ilmeisesti kerätty oppimisprosessin aikana striimaten\cite{carter2015normalized}, Introduction:1:1, 2. Background and related work:2:1
        \begin{itemize}
            \item NPSM keräsi 11 muuttujaa, joista yksi on käytetty aktiivinen aika, 3.2:2:8
            \item näistä kymmenen normalisoitiin tehtävään käytettyyn aikaan, 3.2:1:7
        \end{itemize}
    \item menetelmät ovat koulutuksellinen (educational) datan louhinta ja oppimis analytiikka
    \begin{itemize}
        \item Lue ainakin sisälysluettelot viitteistä 3 ja 24
    \end{itemize}
    \item Ohjelmointi-opetukseen kätevää lisätä data-louhintaa, koska tavoitteena on parantaa opiskelijoiden toistoa, Introduction:2:4
    \item ennusteita tehtiin
    \begin{itemize}
        \item yksittäisten tehtävänantojen arvosanoihin, 4.3.1
        \begin{itemize}
            \item selittävimmät oli tuntematon eli aloitustila, syntaksi virheiden kanssa painimenn ja ohjelman ajaminen ilman koodin korjausta (debugging), 4.3.2:3:3
        \end{itemize}
        \item tehtävien arvosanojen keskiarvoon, 4.3.2
        \begin{itemize}
            \item myöhäinen aloitusvaihe korreloi negatiivisesti, 4.3.2:3:4
            \item syntaksivirheiden kanssa pitkään olo korreloi negatiivisesti, 4.3.2:3:4
        \end{itemize}
        \item Loppuarvosanan enenustaminen, 4.3.3
        \begin{itemize}
            \item tarkemmin, yritettiin selittää varianssi loppuarvosanoissa, 4.3.3:1:1
        \end{itemize}
    \end{itemize}
    \item Mielenkiintoinen huomio optimaalisesta otoskoosta kappaaleessa 5:2:2
    \item Kurssin aikana ennustamista kappaleessa 5
    \item multivariate regression, 5.2:1:1
    \item linear regression, 5.3:3:3
    \item Mallista rakennetut ennustavat funktiot antavat käytettävissä olevia tuloksia, 5.3:3:12
    \item Watwin-asteikko ja Error Quotient perustuvat NPSM mallin vähiten painotettuun merkittävään arvoon: "muokkaa syntaktisesti väärää koodia, edellinen ajo onnistunut", 6.2:3:1
    \begin{itemize}
        \item Lineaari regression ajaminen pelkästään tällä arvolla, NU:lla, "explains more variance" kuin kumpikaan asteikoista
        \begin{itemize}
            \item Vahva epäilys että mittaukset ohjelmointikäytökseen ajamis-käytöksen lisäksi suoriutuvat hyvin
        \end{itemize}
    \end{itemize}
\end{itemize}

\section{Muistiinpanoja: Methods and Tools for Exploring Novice Compilation Behaviour\cite{jadud2006methods}}

Skouppi: Ohjelmoinit kurssimenestyksen ennustaminen

\begin{itemize}
    \item Tarkastelee kääntämis-käyttäytymistä, (compilation behaviour)
    \begin{itemize}
        \item Tarkastelee siis kirjoitettavan koodin syntaktista tilaa
    \end{itemize}
    \item Java kurssille tehty tutkimus
    \begin{itemize}
        \item BlueJ-ympäristö
    \end{itemize}
    \item Kysymyksiä
    \begin{itemize}
        \item Suuria muutoksia session aikana?
        \item Pitkä-kestoisia muutoksia?
        \item ongelmallisia syntaksi-virheitä?
        \item minne muutokset sijoittuivat?
    \end{itemize}
    \item Valitut muuttujat:
    \begin{itemize}
        \item ErrType
        \item Delta-T: ajojen välinen aika
        \item Delta-Ch: muutettujen merkkien määrä
        \item Location: sijainti koodissa
    \end{itemize}
    \item Error Quotient
    \begin{itemize}
        \item Collate: parit tapahtumien välillä sessiossa
        \item Calculate: algoritmin mukainen pisteytys kunkin tapahtuman välillä
        \begin{itemize}
            \item Jos molemmat ovat virheellisiä -> pisteet kasvaa
        \end{itemize}
        \item Normalize: normalisoidaan pistemäärä jakamalla tulos yhdellätoista. Yksitoista on mahdollisten tilojen summa
        \item Average: keskiarvo parien pisteistä sessiossa -> ERROR QUOTIENT
    \end{itemize}
    \item on korrelaatiota opiskelijan EQ:n, loppuarvosanana ja tehtävien arvosanojen keskiarvon välillä on korrelaatiota
    \begin{itemize}
        \item ei kuitenkaa sovi kovinkaan tarkasti: EQ lienee varsin kapea mittari
        \item tarkempi sovitus EQ:n ja loppuarvosanan välillä, edelleen epätarkka
    \end{itemize}
    \item eivät voi tehdä vahvoja väitteitä, että EQ:ta voisi käyttää ennustavana asteikkona tentti-asteikkoon
    \begin{itemize}
        \item Liikaa puuttuvaa tietoa
    \end{itemize}
    \item Paperissa ei mainita mihinkään malliin sovittamista
\end{itemize}


\section{Watson-Score artikkelin muistiinpanoja\cite{watson2013predicting}}

\begin{itemize}
    \item Halutaan tunnistaa vaikeuksissa olevat opiskelijat mahdollisimman varhaisessa vaiheessa
    \begin{itemize}
        \item varhainen opettajan puuttuminen
    \end{itemize}
    \item otos: 45 oppilasta
    \begin{itemize}
        \item yliopiston johdatus ohjelmointiin-kursi
        \item koodi-snapshot jokaisen ajon yhteydessä
    \end{itemize}
    \item pisteytys
    \begin{itemize}
        \item rangaistaan ajasta, joka kuluu tietyn ongelman ratkaisemiseen
        \item verrokkina muut opiskelijat
        \item normalisoituna arvo nollan ja yhden välillä
    \end{itemize}
    \item Käytännösssä kyseessä on tilakone: tarkastellaan perättäisten tilojen vuorovaikutusta
    \item parit muodostettu perättäisistä ajoista ja tiloista tiedosto kohtaisesti. Kronologisessa järjestyksessä hukkuu tieto useamman tiedoston välisistä tiloista
    \item identtiset, perättäiset tilat poistettu IDE:n ominaisuuden vuoksi
    \item poisto-korjaukset suodatettu
    \item virheviestien yleistys
    \item työaikaarviointiin liittyi ongelmia, koska parit on muodostettu tiedostokohtaisesti. Aika käytetty kuhunkin tiedostoon arvioitu koko tapahtumajonosta
    \item soveltuvat ennustavat ominaisuudet
    \begin{itemize}
        \item peräkkäiset virheet olivat
        \begin{itemize}
            \item virheellisiä yleensä
            \item virhetyypit ovat samanlaisia
            \item virhetyypit erilaiset
            \item sama virhesijainti
        \end{itemize}
        \item keskiarvoinen virheen suoritus/selvitysaika. Huomioitiin että eri virheet vaativat eri aikamäärän
        \begin{itemize}
            \item Jälleen yleisen virhetyypin mukaan! Vertailu tapahtuu yleisen virhetyypin mukaan ja vertaillen muihin opiskelijoihin.
        \end{itemize}
        \item rangaistus-pisteet valittiin brute-force haulla, reilun pisteytyksen saavuttamiseksi. Testattiin cross-validation-menetelmällä
    \end{itemize}
    \item arviointi
    \begin{itemize}
        \item lineaari-regresio
        \begin{itemize}
            \item opiskelijoiden Watwin-pisteytys itsenäinen muuttuja
            \item opiskelijoiden kurssin tehtävien pisteytys koko kurssin ajalta
        \end{itemize} 
        \item Harkittiin luokittelijaa kyseisen yliopiston arvosana-asteikon mukaan
        \item on lineaarinen relaatio Watwin-pisteytyksen ja suorituksen välillä, ilman merkittäviä poikkeavuuksia
    \end{itemize}
\end{itemize}

\chapter{Berginin ohjelmointiennustamis-artikkelin muistiinpanot}

Berginin artikkeli koneoppimisen käyttämisestä\cite{bergin2015using}

\begin{itemize}
    \item Motivaatio
    \begin{itemize}
        \item ohjelmoinnin oppimisessa vaikeuksia: korkea pudotusaste
        \item vaikeuksissa olevan opiskelijan tunnistaminen on vaikeaa ajoissa suuresta opiskelija määrästä johtuen.
        \begin{itemize}
            \item ajoitus tärkeä! Liian myöhään, niin opiskelija ei voi jättäytyä pois, tai ohjaajan tuki on myöhässä
        \end{itemize}
        \item edeltävissä tutkimuksissa ongelmia
        \begin{itemize}
            \item tarkat parametrit: biased findings
            \item validaatio-tutkimuksista puuteta
            \item parhaimmissa lähinnä tilastollisia menetelmiä
            \begin{itemize}
                \item rajoittuneet näiden mallien oletuksiin -> ei välttämättä parhain malli
            \end{itemize}
        \end{itemize}
    \end{itemize}
    \item Principal Component Analysis (PCA) ulottuvuuksien pienentämiseen
    \item paperin tarkoituksena on ENNUSTAA ennen kurssin alkua, että miten kurssi etenee
    \begin{itemize}
        \item Tarkemmat kuvaukset arvoista löytyy artikkelin neljännestä viitteestä
        \item etuna on että ei tarvita tietoa kurssityöstä, saati tarkkaa mallia tällaisen tiedon käyttämiseen
    \end{itemize}
    \item valittujen mallien vaatimuksena on että ne ovat toteutettavissa!
    \item testatut koneoppimis-algoritmit
    \begin{itemize}
        \item logistic regression
        \item k-nearest neighbor
        \item backpropagation
        \item C4.5
        \begin{itemize}
            \item Päättelypuu-algoritmi
        \end{itemize}
        \item naive Bayes
        \item support vector machine
    \end{itemize}
    \item Naive Bayes vahvin tarkkuudessa ja herkkyydessä (accuracy, sensitivity)
\end{itemize}

\chapter{Loppusanat\label{chapter:Loppusanat}}

It is good to conclude with some insightful discussion. 

% STEP 5:
% Uncomment the following lines and set your .bib file and desired bibliography style
% to make a bibliography with BibTeX.
% Alternatively you can use the thebibliography environment if you want to add all
% references by hand.

\cleardoublepage %fixes the position of bibliography in bookmarks
\phantomsection

\addcontentsline{toc}{chapter}{\bibname} % This lines adds the bibliography to the ToC
\bibliographystyle{abbrv} % numbering alphabetic order
\bibliography{bibliography}

\begin{appendices}
\myappendixtitle

\chapter{Code example\label{appendix:code}}
Program code can be added as appendix:
\begin{verbatim}
#!/bin/bash          
text="Hello World!"
echo $text
\end{verbatim}

\end{appendices}

\end{document}
