%% This file is modified by Veli Mäkinen from HY_fysiikka_LuKtemplate.tex authored by Roope Halonen ja Tomi Vainio.
%% Some text is also inherited from engl_malli.tex by Kutvonen, Erkiö, Mäkelä, Verkamo, Kurhila, and Nykänen.


% STEP 1: Choose oneside or twoside
\documentclass[finnish,twoside,openright]{HYgraduMLDS}
%finnish,swedish

%\usepackage[utf8]{inputenc} % For UTF8 support. Use UTF8 when saving your file.
\usepackage{lmodern} % Font package
\usepackage{textcomp} % Package for special symbols
\usepackage[pdftex]{color, graphicx} % For pdf output and jpg/png graphics
\usepackage[pdftex, plainpages=false]{hyperref} % For hyperlinks and pdf metadata
\usepackage{fancyhdr} % For nicer page headers
\usepackage{tikz} % For making vector graphics (hard to learn but powerful)
%\usepackage{wrapfig} % For nice text-wrapping figures (use at own discretion)
\usepackage{amsmath, amssymb} % For better math
%\usepackage[square]{natbib} % For bibliography
\usepackage[footnotesize,bf]{caption} % For more control over figure captions
\usepackage{blindtext}
\usepackage{titlesec}
\usepackage[titletoc]{appendix}
\usepackage{hyperref}

\onehalfspacing %line spacing
%\singlespacing
%\doublespacing

%\fussy 
\sloppy % sloppy and fussy commands can be used to avoid overlong text lines

% STEP 2:
% Set up all the information for the title page and the abstract form.
% Replace parameters with your information.
\title{SQL:n Oppiminen Datatieteen Menetelmin}
\author{Matti Räty}
\date{\today}
\prof{Professor X or Dr. Y}
\censors{Professor A}{Dr. B}{}
\keywords{layout, summary, list of references}
\depositeplace{}
\additionalinformation{}


\classification{\protect{\ \\
\  General and reference $\rightarrow$ Document types  $\rightarrow$ Surveys and overviews\  \\
\  Applied computing  $\rightarrow$ Document management and text processing  $\rightarrow$ Document management $\rightarrow$ Text editing\\
}}

% if you want to quote someone special. You can comment this line and there will be nothing on the document.
%\quoting{Bachelor's degrees make pretty good placemats if you get them laminated.}{Jeph Jacques} 


% OPTIONAL STEP: Set up properties and metadata for the pdf file that pdfLaTeX makes.
% But you don't really need to do this unless you want to.
\hypersetup{
    bookmarks=true,         % show bookmarks bar first?
    unicode=true,           % to show non-Latin characters in Acrobat’s bookmarks
    pdftoolbar=true,        % show Acrobat’s toolbar?
    pdfmenubar=true,        % show Acrobat’s menu?
    pdffitwindow=false,     % window fit to page when opened
    pdfstartview={FitH},    % fits the width of the page to the window
    pdftitle={},            % title
    pdfauthor={},           % author
    pdfsubject={},          % subject of the document
    pdfcreator={},          % creator of the document
    pdfproducer={pdfLaTeX}, % producer of the document
    pdfkeywords={something} {something else}, % list of keywords for
    pdfnewwindow=true,      % links in new window
    colorlinks=true,        % false: boxed links; true: colored links
    linkcolor=black,        % color of internal links
    citecolor=black,        % color of links to bibliography
    filecolor=magenta,      % color of file links
    urlcolor=cyan           % color of external links
}

\begin{document}

% Generate title page.
\maketitle

% STEP 3:
% Write your abstract (of course you really do this last).
% You can make several abstract pages (if you want it in different languages),
% but you should also then redefine some of the above parameters in the proper
% language as well, in between the abstract definitions.

\begin{abstract}
Summary of the main contents of the work: topic, methodology and results.

Topics are classified according to the ACM Computing Classification System
(CCS): check command \verb+\classification{}+. A small set of paths (1-3) should be used, starting from any top nodes
referred to bu the root term CCS leading to the leaf nodes. The elements
in the path are separated by right arrow, and emphasis of each element individually can be indicated
by the use of bold face for high importance or italics for intermediate
level. The combination of individual boldface terms may give the reader
additional insight. 
\end{abstract}

% Place ToC
\mytableofcontents

\mynomenclature

% -----------------------------------------------------------------------------------
% STEP 4: Write the thesis.
% Your actual text starts here. You shouldn't mess with the code above the line except
% to change the parameters. Removing the abstract and ToC commands will mess up stuff.
\chapter{Johdanto}

The thesis should have an introduction chapter. Other chapters can be named according to the topic. In the end, some summary chapter is needed; see Chapter~\ref{chapter:Loppusanat} for an example.

\section{Tutkimuskysymykset}

\begin{itemize}
    \item Miten SQL:n oppimista on tutkittu?
    \begin{itemize}
        \item kirjallisuus
    \end{itemize}
    \item Millä tavoin kurssin tehtävien perusteella ennustaa oppimenestystä?
    \begin{itemize}
        \item Kirjallisuus
    \end{itemize}
    \item Millä tavoin SQL-kurssin oppimenestystä ennustaa kurssin tehtävien perusteella?
    \begin{itemize}
        \item kirjallisuus
        \item tämä data
    \end{itemize}
\end{itemize}


\chapter{Taustaa}

\section{SQL:n oppiminen}

Tietotekniikka-alalla SQL-opinnot ovat kuuluneet IT-alan opintoihin vuosikymmenien ajan. Tietokantakieli SQL on nykyajan standardi tekniikka pitkäaikaiseen ja saavutettavissa olevan tiedon tallentamiseen sekä käsittelyyn. Pitkään historaan nähden, SQL:n opetus on saanut varsin vähäistä huomiota tieteellisissä piireissä\cite{Taipalus:2019:EFS:3287324.3287359}. Esimerkiksi aiheesta että mikä on SQL:n opiskelussa vaikeaa. Kyseisen alueen tuloksen ovat tulkinnanvaraisia\cite{Taipalus:2019:EFS:3287324.3287359}.

Perinteisesti SQL:n opetuksen tutkimus on kohdistunut opetustyökaluihin ja opiskelijoiden tekemiin virheisiin\cite{Taipalus:2019:EFS:3287324.3287359}. Opetustyökaluihin lukeutuu usein erilaiset opetusohjlemistot.


\subsection{SQL opetustyökalut}

Tutkimus SQL opetuskaluihin nousennee tutkijoiden kiinnostuksesta e-oppimiseen. Se on ollut suosittu tutkimuksen aihe vuodesta 2000 alkaen. E-oppimista perustellaan mahdollisuudella oppia aktiivisesti tekemällä, sen sijaan että opiskellaan asioita passiivisesti lukemalla. Kehitettyjen työkalujen on todettu parantavan opiskelijoiden kurssiarvosanaa\cite{Brusilovsky:2010:LSP:1656255.1656257}, tosin tilastoja verrokeista ei ole annettu.

E-oppimistyökaluissa yleensä on olennaista käyttötarkoituksen mukainen suunnittelu. Muun muassa luonteva navigointi opetusympäristössä parantaa merkittävästi opetustyökalun käyttöastetta ja sen tarjoamaa hyötyä\cite{Brusilovsky:2010:LSP:1656255.1656257}.

SQL-kyselykielen opetteluun on olemassa useita opetustyökaluja. Yksittäinen opetustyökalu ei yksinään välttämättä tarjoa opetuksen vaatimia ominaisuuksia. Integroida olemassa olevia työkaluja, joissa tuodaa useampi opetusympäristö yhteen käyttöliittymään. Tutkimus olemassa olevien opetustyökalujen yhdistämiseen on vähäistä, ilmeisesti integroitujen työkalujen teknisestä vaikeudesta johtuen\cite{Brusilovsky:2010:LSP:1656255.1656257}.

Integroitujen työkalujen helppokäyttöisyyden varmistamiseksi on Brislovin ym. \cite{Brusilovsky:2010:LSP:1656255.1656257} mukaan mahdollistettava single sign-on, opiskelijoiden toimintojen seuraaminen ja tallennettujen tietojen saatavuus. Tallennetun tiedon olisi tärkeää olla muodossa josta opiskelijoiden osaamista on mahdollista päätellä.


\subsection{Tutkimusta SQL:n oppimisesta / Opiskelijoiden virheet SQL:ssä}

Taipalus ym. \cite{Taipalus:2019:EFS:3287324.3287359} tekivät tutkimusta opiskelijoiden tekemistä virheistä SQL-opetuskurssilla. Data kerättiin heidän tekemässä verkkosovelluksessa, jossa tavoite-taulu oli näkyvillä koko tehtävän tekemisen ajan. Kun tehtävän tekijä oli yritysten jälkeen tyytyväinen kyselyynsä, hän palautti vastauksen. Eniten heitä kiinnosti pysyvät virheet (persistent error), eli virheet jotka esiintyivät palautetuissa tehtävissä. 

Taipalus ym. \cite{Taipalus:2019:EFS:3287324.3287359} jakoi tehdyt virheet neljään yläluokkaan: sekaannuksiin (complication), loogisiin-, semanttisiin- sekä syntaktisiin virheisiin. Sekaannukset eivät vaikuta itse tulostauluun. % tsekkaa niitten toinen artikkeli, jos sais enemmän irti
Loogiset virheet vaikuttavat tulostauluun, mutta tavalla jolla sen tunnistamiseen vaaditaan käsitys ja ymmärrys kyselyn taivoitteesta. Loogisia virheitä ovat esimerkiksi määritelmävirheet (expression error) tai funktiovirheet. Virheet joissa on loogisia virheitä voivat olla hyödyllisiä johonkin toiseen käyttötarkoitukseen. Semanttiset virheet myös vaikuttavat tulostauluun, mutta niiden tunnistaminen on ilmeisempää esimerkiksi toistuvista riveistä. Semanttisesti virheellisillä kyselyillä ei ole käyttökohdetta. Kyselyt joissa on syntaksisia virheitä palauttavat virheilmoituksen ilman vastaustaulua. Syntaktisiin virheisiin lukeutuu esimerkiksi laiton rymittelyfunktio (aggregation-function) tai määrittelemätön tietokantaobjekti.

Yleisin virhe kaikissa SQL kyselyissä ovat syntaktiset virheet\cite{Taipalus:2019:EFS:3287324.3287359}. Luonnollisesti syntaktisten virheiden määrä oli suurempi kurssin alussa, ja vähenivät kurssin edetessä. Yleisimmät pysyvien virheiden luokat ovat puolestaan loogiset virheet ja sekaannukset\cite{Taipalus:2019:EFS:3287324.3287359}. Loogiset virheet ovat vaikeita ratkaista, ja tämä oli tapaus myös Taipaluksen ym. \cite{Taipalus:2019:EFS:3287324.3287359} mainitsemalla SQL kurssilla. Loogisten virheiden vaikeuden on ehdotettu juontuvat ihmisen työmuistin luonteeseen, ja tapaan jolla ihminen kääntää ajatuksiaan SQL:ään\cite{SMELCER1995353}. 

Tehtävät joiden ratkaisut vaativat jokaista kuutta SQL-kielen lauseketta (SQL clause) tuottivat vaikeuksia\cite{Taipalus:2019:EFS:3287324.3287359}. Tavallisesti tehtävissä käytettiin kolmesta neljää eri lauseketta. Jotkin käsitteet ja konseptit toivat mukaansa itsellensä tyypillisiä virheitä mukaan tilastoihin. Esimerkiksi monitauluisissa kyselyissä esiintyi JOIN-kyselyille tyypillisiä syntaktisia- ja semanttisia virheitä. Yksittäinen yleisin ja pysyvin virhetyyppi nousi kokoamisfunktioiden (aggregation function) käytöstä, aina kun tehtävässä niitä tarvittiin. 


\section{Kurssi-menestyksen ennustaminen ohjelmointikursseilla}

Ohjelmointikurssit joiden tarkoitus on olla opiskelijoiden ensimmäinen kosketus ohjelmointiin ovat vaikeita huomattavalle osuudelle opiskelijoista. Tämä johtaa kursseilta pois jäämiseen ja hylättyihin arvosanoihin sekä heikkoihin arvosanoihin \cite{bergin2015using}. Aloittelijoiden ohjelmointikurssien läpipääsymäärä tutkimusaiheena on kasvattanut kasvattanut merkittävästi suosiotaan merkittävästi vuodesta 2009 alkaen\cite{hellas2018predicting}. Kiinnostuksen kasvusta huolimatta maailmanlaajuinen arvioitu läpipääsyjen määrä ohjelmoinnin aloituskursseilla oli noin 68\% vuonna 2014\cite{watson2014failure}. 

Vaikka halua opiskelijoiden auttamiseen onkin, on apua tarvitsevien löytäminen työlästä ja aikaa vievää. Ehdotettuja ratkaisuja ovat muun muassa ennakkoon tehtävät kyselyt\cite{watson2014no}. % Voi vaatia tarkemman sitaatin
Kyselyt ovat kuitenkin hyödyttömiä opiskelijamäärien ollessa yli sata jokaista kurssiohjaajaa kohden: lomakkeiden läpikäyminen vie liikaa aikaa, ja näin apua tarvitsevat löydetään liian myöhään. Tukea tarvitaan mahdollisimman varhaisessa vaiheessa jotta arvosanoihin voidaan vaikuttaa merkittävällä tavalla\cite{bergin2015using}. 

Ongelmaa on lähdetty ratkaisemaan automatisoimalla tunnistamisprosessia. Tunnistamisen autimatisointia on lähdetty ratkaisemaan pyrkimällä ennustamaan opiskelijoiden suoriutumista. Ohjelmointikurssien ennustamisen kohteena ovat usein kurssin loppuarvosana, tenttiarvosana sekä tehtävien kurssiarvosana. Yleisimpiä ennustamismenetelmiä ovat tilastolliset menetelmät\cite{hellas2018predicting}. Muita ohjelmointikurssien yhteydessä käytettäviä menetelmiä ovat muun muassa datalouhimis- ja koneoppimismenetelmät. Ohjelmointikursseilla data on usein kerätty ohjelmointiympäristön (IDE, Integrated Development Environment) keräämästä tiedosta. Tähän voi sisältyä ohjelmakoodin kääntämisen aikana kerätyt kirjaukset (ts. lokitukset), tai tilannekuvat (snapshot) koko ohjelmakoodista\cite{watson2013predicting, jadud2006methods, lagus2018transfer}.

Tilastollisten menetelmien yhteydessä piirteiden mallintamiseen on muutamassa tapauksessa käytetty tilakoneita. Näistä viitatuimpia ovat Error Quotient\cite{jadud2006methods}, Watson Score\cite{watson2013predicting} sekä Normalized Programming State Model (NPSM)\cite{carter2015normalized}. 

Error Quotient ja Watson Score rakentavat mallinsa oman ohjelmointikielen kääntäjän kääntämis-lokeista, ja näin pyrkivät ymmärtämään aloittelevien ohjelmoijien ohjelmointikääntämiskäyttäytymistä. Nämä mallit tarkastelevat opiskelijoiden virheellisiä sovelluksien kääntämisyrityksiä, ja vetävät johtopäätöksiä perättäisten kääntämiskertojen välillä. Error quotient on näistä kahdesta yksinkertaisempi. Se lähinnä tarkastelee perättäisten virheiden tyyppiä. Perättäisten virheiden lisäksi Watson Score ottaa käyttöönsä useampia parametreja. Watson Scoren:n käyttämä pääpiirre on virheiden korjaamiseen kuluva aika. Tätä aikaa verrataan samalla kurssilla oleviin opiskelijoihin, ja pisteytetään tämän mukaan. Jos aikaa kului enemmän kuin vertaisilla, lasketaan kyseisen opiskelijan pisteytystä.

NPSM lienee ottanut innoitusta Watson Score:sta. Kuten Watson Score, myös NPSM tarkkailee kulunutta aikaa. NPSM käyttää kuitenkin tarkempaa mallia ja käyttää ajankohtaisempaa dataa. Mallissa otetaan huomioon, oliko ohjelmakoodissa semanttisia- tai syntaktisia virheitä. Näin NPSM muodostaa holistisemman kuvan kunkin opiskelijan ohjelmakoodin tilasta. NPSM hyödynsi kääntämislokien lisäksi striimattuja otoksia opiskelijoiden koodista, joita opiskelijoiden ohjelmointiympäristät lähettivät, antaen tarkempaa tietoa esimerkiksi ohjelmakoodin syntaktisesta tilasta. 

Ennustavina menetelminä heikoin on Error Quotient, ja selkeästi vahvin on NPSM huomioitaessa selittävää varianssia\cite{carter2015normalized}. Error Quotient-mallista puuttuu liikaa tietoa ollakseen enemmän kuin suuntaa antava pisteytys\cite{jadud2006methods}. Vaikka Watson Score on tarkempi Error Quotient:iin nähden, on sen käyttämä data paljon suppeampaa verrattuna NPSM:iin\cite{carter2015normalized}. Carter kuitenkin huomautti mahdollisista eroista kurssiasetelmassa, ohjelmistoympäristöissä ja ohjelmointikielissä\cite{carter2015normalized}. 

Kullakin tilakoneella rakennettiin ennustava malli lineaarisella regressiolla. NPSM:a sovellettiin myös multivariate regressioon. Mielenkiintoinen tutkimisen aihe olisi soveltaa näiden tilakoneiden tuottamia parametreja monimutkaisempiin koneoppimismalleihin.

Opiskelijoiden ohjelmointikurssien menestyksen kontekstiin sovellettuja koneoppimistekniikoita on laaja kirjo. Kun kyseessä on akateemisen menestyksen ennustaminen, ovat suosituimpia tekniikoita lineaarinen regressio, ja luokittelualgoritmit\cite{hellas2018predicting}. Tämä heijastuu myös ohjelmointikurssien menestyksen ennustamiseen: tarkimpien menetelmien joukkoon kuuluvat muun muassa Naive Bayes\cite{bergin2015using}, tukivektorikoneet\cite{bergin2015using} sekä random forest\cite{lagus2018transfer}. Näistä jokainen on luokittelukoneoppmis-tekniikka. Neuroverkot ovat myös merkittävä tutkimuksen kohde ohjelmointikurssien menestyksen ennustamisessa\cite{Castro-Wunsch:2017:ENN:3017680.3017792}. 

Koneoppimisalgoritmien ongelmana on usein niiden joustamattomuus: ne on koulutettava datasta, joka on samanlaista kuin itse käytössä oleva data. Tämä pätee erityisesti klassisiin koneoppimisalgoritmeihin kuten esimerkiksi tukivektorikoneisiin ja logistiseen regressioon. Jos tätä ehtoa ei noudateta, ovat ennusteiden tarkkuudet huomattavan heikkoja. Jos koneoppimisalgoritmi koulutetaan kurssia varten ja kurssin aikana, todennäköisesti ongelmaa ei ole. Tavoiteltavaa on kuitenkin saada aikaiseksi malli, jolla on kohtuullinen tarkkuus edes saman kurssin eri toistokerroilla. Vaikka ohjelmointikurssilla kurssimateriaali, tavoitteet ja opeteltavat asiat pysyisivät samana vuodesta toiseen, on muuttuvia tekijöitä jokaisella iteraatiokerralla useita. Muun muassa osallistujien lähtötaso, opettajat sekä opetusmenetelmät. Muuttuvat tekijät pudottavat ennusteiden tarkkuutta merkittävästi. Saman kurssin eri toistoihin sovellettuja algoritmeja on tutkittu muun muassa neuroverkoilla\cite{Castro-Wunsch:2017:ENN:3017680.3017792} ja transfer learning\cite{lagus2018transfer}-tekniikalla.


\section{SQL-kurssien menestyksen ennustaminen}

Ainakin kolme artikkelia aiheesta.


\chapter{Koneoppimisen soveltaminen kurssi-dataan}

\section{Data}

Mitä dataa, mistä on saatu ja suunnitelmia käytöstä.


\section{Valitut mentelmät}


\chapter{SQL-kurssin oppimenestyksen ennustamisen toteutus}

\section{Ohjelmakoodi}


\section{Prosessi, haasteet, yms.}


\section{Tulokset}


\chapter{Loppusanat\label{chapter:Loppusanat}}

It is good to conclude with some insightful discussion. 

% STEP 5:
% Uncomment the following lines and set your .bib file and desired bibliography style
% to make a bibliography with BibTeX.
% Alternatively you can use the thebibliography environment if you want to add all
% references by hand.

\cleardoublepage %fixes the position of bibliography in bookmarks
\phantomsection

\addcontentsline{toc}{chapter}{\bibname} % This lines adds the bibliography to the ToC
\bibliographystyle{abbrv} % numbering alphabetic order
\bibliography{bibliography}

\begin{appendices}
\myappendixtitle

\chapter{Code example\label{appendix:code}}
Program code can be added as appendix:
\begin{verbatim}
#!/bin/bash          
text="Hello World!"
echo $text
\end{verbatim}

\end{appendices}

\end{document}
