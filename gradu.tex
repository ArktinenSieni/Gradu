%% This file is modified by Veli Mäkinen from HY_fysiikka_LuKtemplate.tex authored by Roope Halonen ja Tomi Vainio.
%% Some text is also inherited from engl_malli.tex by Kutvonen, Erkiö, Mäkelä, Verkamo, Kurhila, and Nykänen.


% STEP 1: Choose oneside or twoside
\documentclass[finnish,twoside,openright]{HYgraduMLDS}
%finnish,swedish

%\usepackage[utf8]{inputenc} % For UTF8 support. Use UTF8 when saving your file.
\usepackage{lmodern} % Font package
\usepackage{textcomp} % Package for special symbols
\usepackage[pdftex]{color, graphicx} % For pdf output and jpg/png graphics
\usepackage[pdftex, plainpages=false]{hyperref} % For hyperlinks and pdf metadata
\usepackage{fancyhdr} % For nicer page headers
\usepackage{tikz} % For making vector graphics (hard to learn but powerful)
%\usepackage{wrapfig} % For nice text-wrapping figures (use at own discretion)
\usepackage{amsmath, amssymb} % For better math
%\usepackage[square]{natbib} % For bibliography
\usepackage[footnotesize,bf]{caption} % For more control over figure captions
\usepackage{blindtext}
\usepackage{titlesec}
\usepackage[titletoc]{appendix}

\onehalfspacing %line spacing
%\singlespacing
%\doublespacing

%\fussy 
\sloppy % sloppy and fussy commands can be used to avoid overlong text lines

% STEP 2:
% Set up all the information for the title page and the abstract form.
% Replace parameters with your information.
\title{SQL:n Oppiminen Datatieteen Menetelmin}
\author{Matti Räty}
\date{\today}
\prof{Professor X or Dr. Y}
\censors{Professor A}{Dr. B}{}
\keywords{layout, summary, list of references}
\depositeplace{}
\additionalinformation{}


\classification{\protect{\ \\
\  General and reference $\rightarrow$ Document types  $\rightarrow$ Surveys and overviews\  \\
\  Applied computing  $\rightarrow$ Document management and text processing  $\rightarrow$ Document management $\rightarrow$ Text editing\\
}}

% if you want to quote someone special. You can comment this line and there will be nothing on the document.
%\quoting{Bachelor's degrees make pretty good placemats if you get them laminated.}{Jeph Jacques} 


% OPTIONAL STEP: Set up properties and metadata for the pdf file that pdfLaTeX makes.
% But you don't really need to do this unless you want to.
\hypersetup{
    bookmarks=true,         % show bookmarks bar first?
    unicode=true,           % to show non-Latin characters in Acrobat’s bookmarks
    pdftoolbar=true,        % show Acrobat’s toolbar?
    pdfmenubar=true,        % show Acrobat’s menu?
    pdffitwindow=false,     % window fit to page when opened
    pdfstartview={FitH},    % fits the width of the page to the window
    pdftitle={},            % title
    pdfauthor={},           % author
    pdfsubject={},          % subject of the document
    pdfcreator={},          % creator of the document
    pdfproducer={pdfLaTeX}, % producer of the document
    pdfkeywords={something} {something else}, % list of keywords for
    pdfnewwindow=true,      % links in new window
    colorlinks=true,        % false: boxed links; true: colored links
    linkcolor=black,        % color of internal links
    citecolor=black,        % color of links to bibliography
    filecolor=magenta,      % color of file links
    urlcolor=cyan           % color of external links
}

\begin{document}

% Generate title page.
\maketitle

% STEP 3:
% Write your abstract (of course you really do this last).
% You can make several abstract pages (if you want it in different languages),
% but you should also then redefine some of the above parameters in the proper
% language as well, in between the abstract definitions.

\begin{abstract}
Summary of the main contents of the work: topic, methodology and results.

Topics are classified according to the ACM Computing Classification System
(CCS): check command \verb+\classification{}+. A small set of paths (1-3) should be used, starting from any top nodes
referred to bu the root term CCS leading to the leaf nodes. The elements
in the path are separated by right arrow, and emphasis of each element individually can be indicated
by the use of bold face for high importance or italics for intermediate
level. The combination of individual boldface terms may give the reader
additional insight. 
\end{abstract}

% Place ToC
\mytableofcontents

\mynomenclature

% -----------------------------------------------------------------------------------
% STEP 4: Write the thesis.
% Your actual text starts here. You shouldn't mess with the code above the line except
% to change the parameters. Removing the abstract and ToC commands will mess up stuff.
\chapter{Johdanto}

The thesis should have an introduction chapter. Other chapters can be named according to the topic. In the end, some summary chapter is needed; see Chapter~\ref{chapter:Loppusanat} for an example.

\section{Tutkimuskysymykset}

\begin{itemize}
    \item Miten SQL:n oppimista on tutkittu?
    \begin{itemize}
        \item kirjallisuus
    \end{itemize}
    \item Millä tavoin kurssin tehtävien perusteella ennustaa oppimenestystä?
    \begin{itemize}
        \item Kirjallisuus
    \end{itemize}
    \item Millä tavoin SQL-kurssin oppimenestystä ennustaa kurssin tehtävien perusteella?
    \begin{itemize}
        \item kirjallisuus
        \item tämä data
    \end{itemize}
\end{itemize}


\chapter{Taustaa}

\section{SQL:n oppiminen}

Ainakin kolme artikkelia aiheesta.

\subsection{SQL Opetusohjelmistot}


\subsection{Tutkimusta SQL:n oppimisesta}


\section{Kurssi-menestyksen ennustaminen ohjelmointikursseilla}

Ainakin kolme artikkelia aiheesta.

\begin{itemize}
    \item Miksi kiinnostavaa
    \item Miten toteutettu, pari keissiä
    \begin{itemize}
        \item Mitä dataa ovat ottaneet
        \item Millä menetelmin
        \item Millaisia tuloksia
    \end{itemize}
\end{itemize}

Akateemisella menestyksellä tarkoitan yleisesti oppimenestystä, ja taitoja mitkä auttavat oppimisessa.

Kiinnostavia aiheita akatemmisen menestyksen ennustamisessa ovat esimerkiksi lisäapua vaativien oppilaiden tunnistaminen ja vaikeiden käsitteiden löytäminen automaattisesti.

Oppimenestyksen ennustamista voidaan tehdä usealla tasolla. Esimerkiksi yksittäistä kurssia, tehtävää tai oppilasta tarkastellen\cite{hellas2018predicting}. Tässä keskitytään yksittäisen kurssin ennustamiseen, joka on pidetty useampana vuonna. Ennustaminen tässä kontekstissa tehdään tehtyjen tehtävien perusteella.

Oppimenestyksen ennustamisen menetelmät ovat jaettavissa luokitteluun, data-louhintaan, tilastollisiin mentelmiin ja klusterointiin\cite{hellas2018predicting}. Käytetyimpiä koneoppimismalleja akateemisen menestyksen ennustamiseen ovat lineaariset regressiomallit ja luokittelumenetelmät. Esimerkkinä monimutkaisemmasta mallista on transferenssi-oppiminen (transfer learning)\cite{lagus2018transfer} ja nönnönnöö.

\subsection{Muistiinpanoja: The Normalized Programming State Model}

\subsubsection{tutkimuskysymys}

\begin{itemize}
    \item Miten eri tavoin eri ympäristöt, kielet ja ohjelmointiympäristöt vaikuttavat Error Quotient\cite{jadud2006methods}- ja Watwin\cite{watson2014no} piste-asteikon ennustaviin kykyihin?
    \item Kuinka hyvin kokonaisvaltaisempi malli ennustaa suoritusta?
\end{itemize}


\subsubsection{tutkimustulokset}

Normalized Programming State Model eli NPSM, suoriutui tutkimuksen\cite{carter2015normalized} ympäristössä Error Qutient- ja Watwinin piste-asteikkoa paremmin.


\subsubsection{muistiinpanoja}

\begin{itemize}
    \item Ennusteet auttavat opettajia tunnistamaan riskirajoilla olevia opiskelijoita, ja parantamaan opetusmenetelmiä\cite{carter2015normalized}, Abstract:1:3
    \item Error Quotient asteikko\cite{jadud2006methods} ja Watwinin pisteasteikko\cite{watson2014no} ovat saavuttaneet menestystä tarkastellen ainoastaan opiskelijoiden kääntämis-yrityksiä(compilation), Abstract:1:6
    \item dataa ilmeisesti kerätty oppimisprosessin aikana striimaten\cite{carter2015normalized}, Introduction:1:1, 2. Background and related work:2:1
        \begin{itemize}
            \item NPSM keräsi 11 muuttujaa, joista yksi on käytetty aktiivinen aika, 3.2:2:8
            \item näistä kymmenen normalisoitiin tehtävään käytettyyn aikaan, 3.2:1:7
        \end{itemize}
    \item menetelmät ovat koulutuksellinen (educational) datan louhinta ja oppimis analytiikka
    \begin{itemize}
        \item Lue ainakin sisälysluettelot viitteistä 3 ja 24
    \end{itemize}
    \item Ohjelmointi-opetukseen kätevää lisätä data-louhintaa, koska tavoitteena on parantaa opiskelijoiden toistoa, Introduction:2:4
    \item ennusteita tehtiin
    \begin{itemize}
        \item yksittäisten tehtävänantojen arvosanoihin, 4.3.1
        \item tehtävien arvosanojen keskiarvoon, 4.3.2
    \end{itemize}
\end{itemize}


\section{SQL-kurssien menestyksen ennustaminen}

Ainakin kolme artikkelia aiheesta.


\chapter{Koneoppimisen soveltaminen kurssi-dataan}

\section{Data}

Mitä dataa, mistä on saatu ja suunnitelmia käytöstä.


\section{Valitut mentelmät}


\chapter{SQL-kurssin oppimenestyksen ennustamisen toteutus}

\section{Ohjelmakoodi}


\section{Prosessi, haasteet, yms.}


\section{Tulokset}


\chapter{Loppusanat\label{chapter:Loppusanat}}

It is good to conclude with some insightful discussion. 

% STEP 5:
% Uncomment the following lines and set your .bib file and desired bibliography style
% to make a bibliography with BibTeX.
% Alternatively you can use the thebibliography environment if you want to add all
% references by hand.

\cleardoublepage %fixes the position of bibliography in bookmarks
\phantomsection

\addcontentsline{toc}{chapter}{\bibname} % This lines adds the bibliography to the ToC
\bibliographystyle{abbrv} % numbering alphabetic order
\bibliography{bibliography}

\begin{appendices}
\myappendixtitle

\chapter{Code example\label{appendix:code}}
Program code can be added as appendix:
\begin{verbatim}
#!/bin/bash          
text="Hello World!"
echo $text
\end{verbatim}

\end{appendices}

\end{document}
