\begin{document}

\begin{itemize}
    \item syntaktiset- ja semanttiset virheet harvinaisempia kuin loogiset virheet ja sekaannukset
    \item SQL on kuulunut yliopistotason opintoihin vuosikymmeniä
    \item tieteellinen näyttö aiheille jotka ovat opiskelijoille vaikeita, on tulkinnanvaraista
    \item tutkimus SQL:n opetuksesta keskittyy joko opinto-työkaluihin ja opiskelijoiden virheisiin SQL:ssä
    \item virheet jaettavissa neljään luokkaan, lähteen 11 mukaan
    \begin{itemize}
        \item complication: eivät vaikuta tulostauluun
        \item logical error: vaikuttaa tulostauluun joka vaikuttaa tulostauluun. Näille tauluille on olemassa pätevä kysyntää
        \begin{itemize}
            \item operaattorivirhe LOG-1
            \item JOIN virhe LOG-2
            \item sisennysvirhe LOG-3
            \item määritelmävirhe LOG-4
            \item projektiovirhe LOG-5
            \item funktiovirhe LOG-6
            \item pakko tietää kysyntä/tarkoitus
        \end{itemize}
        \item semanttinen virhe: vaikuttaa tulostauluun, jolle ei ole pätevää kysyntää
        \begin{itemize}
            \item epäjohdonmukainen määritelmä SEM-1
            \item epäjohdonmukainen JOIN SEM-2
            \item puuttuva JOIN SEM-3
            \item toistuvia rivejä SEM-4
            \item tarpeeton sarake SEM-5
            \item ilmeinen katsomatta kysyntää/tarkoitusta
        \end{itemize}
        \item syntaktinen virhe: palauttaa virheviestin tulostaulun sijasta
        \begin{itemize}
            \item epäselvä tietokanta objekti SYN-1
            \item määrittelemätön tietokantaobjekti SYN-2
            \item yhteensopimaton tietotyyppi SYN-3
            \item laiton ryhmittely(aggregation) funktion asettelu SYN-4
            \item laiton tai riittämätön ryhmittely SYN-5
            \item yleinen syntaksivirhe SYN-6
        \end{itemize}
    \end{itemize}
    \item pysyvimmät virheet
    \begin{itemize}
        \item laiton tai riittämätön ryhmittely SYN-5
        \item yleinen syntaksivirhe SYN-6
        \item epäjohdonmukainen määritelmä SEM-1
        \item epäjohdonmuainen JOIN SEM-2
        \item puuttuva JOIN SEM-3
        \item määritelmävirhe LOG-4
        \item projektiovirhe LOG-6
    \end{itemize}
    \item kaikkien kyselyiden kesken yleisimpiä ovat
    \begin{itemize}
        \item yleinen syntaksivirhe SYN-6
        \item määrittelemätön tietokantaobjekti SYN-2
        \item laiton tai riittämätön ryhmittely SYN-5
    \end{itemize}
    \item loogiset virheet ovat vaikeita ratkaista
    \item viite 10 ehdottaa että loogiset virheet liittyvät lyhytaikaiseen muistiin, ja ajatuksenkulusta mikä liittynee ajatusprosessiin luonnollisenkielenkääntämisestä SQL:n
    \item sekaannukset (complications) ovat yleisiä, mutta yleensä korjaantuvat kyselyn kehittyessä
    \item erilaiset kyselykonseptit tuovat erilaisia virheitä by design
    \begin{itemize}
        \item esimerkiksi kokoamisfunktiot tuovat funktiovirheitä
    \end{itemize}
    \item lienee johdonmukaista olettaa että opettajien ja tutkijoiden pitäisi keskittyä virheisiin jotka ovat yleisiä ja pysyviä
    \item virheelliset datatietotyyppivirheet vähenivät kurssin edetessä
    \item kokoamis (aggregate) funktiot voat vaikeita
    \item kyselyt, jotka vaativat viittä tai kuutta SQL ehtoa (clauses) ovat vaikeita. Yleensä kyselyissä on kolmesta neljään ehtoa
    \item epäjohdonmukaiset määritelmät olivat yleisiä: jokainen kysely sisältää niitä
    \item moni-tauluiset kyselyt olivat yleisiä siellä missä niitä käytettiin.
    \item loogiset virheet ja sekaannukset ovat pysyvämpiä verrattuna syntaktisiin ja semanttisiin virheisiin
    \item funktiovirheet olivat yleisiä tehtävissä tietyillä kyselykonsepteilla
    \item säilyvimmät virheet olivat määritelmä ja projektiovirheet kyselykonsepteista huolimatta
\end{itemize}

\chapter{Muistiinpanoja}

\section{Muistiinpanoja: The Normalized Programming State Model}

Oleellinen termi: explained variance
\begin{itemize}
    \item \url{https://en.m.wikipedia.org/wiki/Explained\_variation}
    \item \url{http://www.csun.edu/~mr31841/documents/Varitionandpredictionintervals.pdf}
\end{itemize}

\subsection{tutkimuskysymys}

\begin{itemize}
    \item Miten eri tavoin eri ympäristöt, kielet ja ohjelmointiympäristöt vaikuttavat Error Quotient\cite{jadud2006methods}- ja Watwin\cite{watson2014no} piste-asteikon ennustaviin kykyihin?
    \item Kuinka hyvin kokonaisvaltaisempi malli ennustaa suoritusta?
\end{itemize}


\subsection{tutkimustulokset}

Normalized Programming State Model eli NPSM, suoriutui tutkimuksen\cite{carter2015normalized} ympäristössä Error Qutient- ja Watwinin piste-asteikkoa paremmin.

\begin{itemize}
    \item RQ1
        \begin{itemize}
            \item rajua vaihtelua mittauksissa Error Quotient ja Watwin-asteikossa, 6.1:1:1
            \begin{itemize}
                \item Tämän kokeen asetelmassa suurempi arvosanapainotus tehtäville, 6.1:2:5
                \item edellämainitut tukeutuvat vahvemmin ohjelmointiympäristön virheviesteihin ja tukeen, ja Javan parempiin virheviesteihin. Verrattuna tämän tutkimuksen kieleen C++:n, 6.1:3:3
                \item Molemmat rankaisevat toistuvista virheviesteistä, 6.1:3:9
                \begin{itemize}
                    \item C++:n virheviestit voivat sisältää samoja virkkeitä eri virheissä -> false positive
                \end{itemize}
            \end{itemize}
            \item Tutkimuksessa toistui aikaisemmat ilmiöt: Error quotient toimii parhaiten pienemmillä tietolähteillä, ja Watwin suuremmilla tietolähteillä, 6.1:5:2
            \item Watwin-asteikon "variance explained" toistuu muissa tutkimuksissa, missä taas Error Quotient:n vastaava arvo ei toistu, 6.1:5:7
            \item Kaikki mallit vaativat lisätutkimuksia eri ympäristöissä, 6.1:5:13
        \end{itemize}
    \item RQ2
    \begin{itemize}
        \item NPSM päihitti mallit jotka keskittyivät vain kääntämis-käyttäytymiseen (compilation), 6.2:1
        \begin{itemize}
            \item Yksittäisissä tehtävissä Watwin asteikkoon nähden nelinkertaiset tulokset varianssin selitettävyydessä, ja Error Quotient:iin nähden prosentin verran
            \item Tehtävä-arvosanojen keskiarvossa 4x Watwin-asteikkoon, ja 6x Error Quotient asteikkoon nähden
            \item Loppuarvosanoijen ennustettavuudessa, 3x Watwin-asteikkoon, ja 12x Error Quotient nähden
        \end{itemize}
        \item Ajaminen semanttisesti väärää ohjelmaa, ja semanttisesti tuntematonta sovellusta edistivät onnistumista, 6.2:2:2
        \item tuntematon tila (eli oletustila ennen ensimmäistä ajoa) ja syntaktisesti väärän ohjelman ajaminen tekivät hallaa onnistumiselle, 6.2:2:6
        \item vetivät johtopäätöksen että sovelluksen ajaminen, oli kyseessä oikea ratkaisu tai ei, vaikuttaa olevan onnistuva ohjelmointi lähestymistapa, 6.2:2:11
        \item koodin kirjoittaminen ilman ajamista isoissa paloissa, päinvastoin, 6.2:2:13
    \end{itemize}
\end{itemize} 


\subsection{muistiinpanoja}

\begin{itemize}
    \item Ennusteet auttavat opettajia tunnistamaan riskirajoilla olevia opiskelijoita, ja parantamaan opetusmenetelmiä\cite{carter2015normalized}, Abstract:1:3
    \item Error Quotient asteikko\cite{jadud2006methods} ja Watwinin pisteasteikko\cite{watson2014no} ovat saavuttaneet menestystä tarkastellen ainoastaan opiskelijoiden kääntämis-yrityksiä(compilation), Abstract:1:6
    \item dataa ilmeisesti kerätty oppimisprosessin aikana striimaten\cite{carter2015normalized}, Introduction:1:1, 2. Background and related work:2:1
        \begin{itemize}
            \item NPSM keräsi 11 muuttujaa, joista yksi on käytetty aktiivinen aika, 3.2:2:8
            \item näistä kymmenen normalisoitiin tehtävään käytettyyn aikaan, 3.2:1:7
        \end{itemize}
    \item menetelmät ovat koulutuksellinen (educational) datan louhinta ja oppimis analytiikka
    \begin{itemize}
        \item Lue ainakin sisälysluettelot viitteistä 3 ja 24
    \end{itemize}
    \item Ohjelmointi-opetukseen kätevää lisätä data-louhintaa, koska tavoitteena on parantaa opiskelijoiden toistoa, Introduction:2:4
    \item ennusteita tehtiin
    \begin{itemize}
        \item yksittäisten tehtävänantojen arvosanoihin, 4.3.1
        \begin{itemize}
            \item selittävimmät oli tuntematon eli aloitustila, syntaksi virheiden kanssa painimenn ja ohjelman ajaminen ilman koodin korjausta (debugging), 4.3.2:3:3
        \end{itemize}
        \item tehtävien arvosanojen keskiarvoon, 4.3.2
        \begin{itemize}
            \item myöhäinen aloitusvaihe korreloi negatiivisesti, 4.3.2:3:4
            \item syntaksivirheiden kanssa pitkään olo korreloi negatiivisesti, 4.3.2:3:4
        \end{itemize}
        \item Loppuarvosanan enenustaminen, 4.3.3
        \begin{itemize}
            \item tarkemmin, yritettiin selittää varianssi loppuarvosanoissa, 4.3.3:1:1
        \end{itemize}
    \end{itemize}
    \item Mielenkiintoinen huomio optimaalisesta otoskoosta kappaaleessa 5:2:2
    \item Kurssin aikana ennustamista kappaleessa 5
    \item multivariate regression, 5.2:1:1
    \item linear regression, 5.3:3:3
    \item Mallista rakennetut ennustavat funktiot antavat käytettävissä olevia tuloksia, 5.3:3:12
    \item Watwin-asteikko ja Error Quotient perustuvat NPSM mallin vähiten painotettuun merkittävään arvoon: "muokkaa syntaktisesti väärää koodia, edellinen ajo onnistunut", 6.2:3:1
    \begin{itemize}
        \item Lineaari regression ajaminen pelkästään tällä arvolla, NU:lla, "explains more variance" kuin kumpikaan asteikoista
        \begin{itemize}
            \item Vahva epäilys että mittaukset ohjelmointikäytökseen ajamis-käytöksen lisäksi suoriutuvat hyvin
        \end{itemize}
    \end{itemize}
\end{itemize}

\section{Muistiinpanoja: Methods and Tools for Exploring Novice Compilation Behaviour\cite{jadud2006methods}}

Skouppi: Ohjelmoinit kurssimenestyksen ennustaminen

\begin{itemize}
    \item Tarkastelee kääntämis-käyttäytymistä, (compilation behaviour)
    \begin{itemize}
        \item Tarkastelee siis kirjoitettavan koodin syntaktista tilaa
    \end{itemize}
    \item Java kurssille tehty tutkimus
    \begin{itemize}
        \item BlueJ-ympäristö
    \end{itemize}
    \item Kysymyksiä
    \begin{itemize}
        \item Suuria muutoksia session aikana?
        \item Pitkä-kestoisia muutoksia?
        \item ongelmallisia syntaksi-virheitä?
        \item minne muutokset sijoittuivat?
    \end{itemize}
    \item Valitut muuttujat:
    \begin{itemize}
        \item ErrType
        \item Delta-T: ajojen välinen aika
        \item Delta-Ch: muutettujen merkkien määrä
        \item Location: sijainti koodissa
    \end{itemize}
    \item Error Quotient
    \begin{itemize}
        \item Collate: parit tapahtumien välillä sessiossa
        \item Calculate: algoritmin mukainen pisteytys kunkin tapahtuman välillä
        \begin{itemize}
            \item Jos molemmat ovat virheellisiä -> pisteet kasvaa
        \end{itemize}
        \item Normalize: normalisoidaan pistemäärä jakamalla tulos yhdellätoista. Yksitoista on mahdollisten tilojen summa
        \item Average: keskiarvo parien pisteistä sessiossa -> ERROR QUOTIENT
    \end{itemize}
    \item on korrelaatiota opiskelijan EQ:n, loppuarvosanana ja tehtävien arvosanojen keskiarvon välillä on korrelaatiota
    \begin{itemize}
        \item ei kuitenkaa sovi kovinkaan tarkasti: EQ lienee varsin kapea mittari
        \item tarkempi sovitus EQ:n ja loppuarvosanan välillä, edelleen epätarkka
    \end{itemize}
    \item eivät voi tehdä vahvoja väitteitä, että EQ:ta voisi käyttää ennustavana asteikkona tentti-asteikkoon
    \begin{itemize}
        \item Liikaa puuttuvaa tietoa
    \end{itemize}
    \item Paperissa ei mainita mihinkään malliin sovittamista
\end{itemize}


\section{Watson-Score artikkelin muistiinpanoja\cite{watson2013predicting}}

\begin{itemize}
    \item Halutaan tunnistaa vaikeuksissa olevat opiskelijat mahdollisimman varhaisessa vaiheessa
    \begin{itemize}
        \item varhainen opettajan puuttuminen
    \end{itemize}
    \item otos: 45 oppilasta
    \begin{itemize}
        \item yliopiston johdatus ohjelmointiin-kursi
        \item koodi-snapshot jokaisen ajon yhteydessä
    \end{itemize}
    \item pisteytys
    \begin{itemize}
        \item rangaistaan ajasta, joka kuluu tietyn ongelman ratkaisemiseen
        \item verrokkina muut opiskelijat
        \item normalisoituna arvo nollan ja yhden välillä
    \end{itemize}
    \item Käytännösssä kyseessä on tilakone: tarkastellaan perättäisten tilojen vuorovaikutusta
    \item parit muodostettu perättäisistä ajoista ja tiloista tiedosto kohtaisesti. Kronologisessa järjestyksessä hukkuu tieto useamman tiedoston välisistä tiloista
    \item identtiset, perättäiset tilat poistettu IDE:n ominaisuuden vuoksi
    \item poisto-korjaukset suodatettu
    \item virheviestien yleistys
    \item työaikaarviointiin liittyi ongelmia, koska parit on muodostettu tiedostokohtaisesti. Aika käytetty kuhunkin tiedostoon arvioitu koko tapahtumajonosta
    \item soveltuvat ennustavat ominaisuudet
    \begin{itemize}
        \item peräkkäiset virheet olivat
        \begin{itemize}
            \item virheellisiä yleensä
            \item virhetyypit ovat samanlaisia
            \item virhetyypit erilaiset
            \item sama virhesijainti
        \end{itemize}
        \item keskiarvoinen virheen suoritus/selvitysaika. Huomioitiin että eri virheet vaativat eri aikamäärän
        \begin{itemize}
            \item Jälleen yleisen virhetyypin mukaan! Vertailu tapahtuu yleisen virhetyypin mukaan ja vertaillen muihin opiskelijoihin.
        \end{itemize}
        \item rangaistus-pisteet valittiin brute-force haulla, reilun pisteytyksen saavuttamiseksi. Testattiin cross-validation-menetelmällä
    \end{itemize}
    \item arviointi
    \begin{itemize}
        \item lineaari-regresio
        \begin{itemize}
            \item opiskelijoiden Watwin-pisteytys itsenäinen muuttuja
            \item opiskelijoiden kurssin tehtävien pisteytys koko kurssin ajalta
        \end{itemize} 
        \item Harkittiin luokittelijaa kyseisen yliopiston arvosana-asteikon mukaan
        \item on lineaarinen relaatio Watwin-pisteytyksen ja suorituksen välillä, ilman merkittäviä poikkeavuuksia
    \end{itemize}
\end{itemize}

\chapter{Berginin ohjelmointiennustamis-artikkelin muistiinpanot}

Berginin artikkeli koneoppimisen käyttämisestä\cite{bergin2015using}

\begin{itemize}
    \item Motivaatio
    \begin{itemize}
        \item ohjelmoinnin oppimisessa vaikeuksia: korkea pudotusaste
        \item vaikeuksissa olevan opiskelijan tunnistaminen on vaikeaa ajoissa suuresta opiskelija määrästä johtuen.
        \begin{itemize}
            \item ajoitus tärkeä! Liian myöhään, niin opiskelija ei voi jättäytyä pois, tai ohjaajan tuki on myöhässä
        \end{itemize}
        \item edeltävissä tutkimuksissa ongelmia
        \begin{itemize}
            \item tarkat parametrit: biased findings
            \item validaatio-tutkimuksista puuteta
            \item parhaimmissa lähinnä tilastollisia menetelmiä
            \begin{itemize}
                \item rajoittuneet näiden mallien oletuksiin -> ei välttämättä parhain malli
            \end{itemize}
        \end{itemize}
    \end{itemize}
    \item Principal Component Analysis (PCA) ulottuvuuksien pienentämiseen
    \item paperin tarkoituksena on ENNUSTAA ennen kurssin alkua, että miten kurssi etenee
    \begin{itemize}
        \item Tarkemmat kuvaukset arvoista löytyy artikkelin neljännestä viitteestä
        \item etuna on että ei tarvita tietoa kurssityöstä, saati tarkkaa mallia tällaisen tiedon käyttämiseen
    \end{itemize}
    \item valittujen mallien vaatimuksena on että ne ovat toteutettavissa!
    \item testatut koneoppimis-algoritmit
    \begin{itemize}
        \item logistic regression
        \item k-nearest neighbor
        \item backpropagation
        \item C4.5
        \begin{itemize}
            \item Päättelypuu-algoritmi
        \end{itemize}
        \item naive Bayes
        \item support vector machine
    \end{itemize}
    \item Naive Bayes vahvin tarkkuudessa ja herkkyydessä (accuracy, sensitivity)
\end{itemize}

\begin{itemize}
    \item e-oppiminen on ollut suosittu tutkimuksen kohde vuodesta 2000 eteenpäin\cite{Brusilovsky:2010:LSP:1656255.1656257}
    \item opetusohjelmistojen tukijat toivovat siirtymistä passiivisesta ja tehottomasta lukemalla oppimisesta, aktiiviseen tekemällä oppimiseen\cite{Brusilovsky:2010:LSP:1656255.1656257}
    \item olemassa olevien työkalujen integroiminen ja personoiminen on ollut vähäistä teknisten vaikeuksien vuoksi
    \item opiskelijoiden etenemisen seuraaminen hankalaa olemassa olevissa työkaluissa
    \item integroidun järjestelmän on mahdollistettava työkalun käyttö ilman montaa kirjautumista
    \item integroidun järjestelmän on pystyttävä seuraamaan opiskelijoiden toimintoja (kaikista mukana olevista järjestelmistä) ja tallennettuna siten että muut järjestelmät voivat hyödyntää niitä
    \item integroidun järjestelmän on mahdollistettava opiskelijoiden tiedon päätteleminen heidän kirjatuista toiminnoistaan eri järjestelmissä
    \item käytetyt työkalut parantavat tehtäväpalautukisen arvosanoja
    \begin{itemize}
        \item ei kyselyä ryhmälle, joka ei käyttänyt työkalua
    \end{itemize}
    \item Työkalua suunnitellessa käyttötarkoituksen mukainen työkalun personointi on tärkeää: vahvistaa e-oppimisen hyötyjä. Muun muassa luonteva ohjelmistossa suunnistaminen on tärkeää
\end{itemize}

\end{document}
